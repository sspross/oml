%
%  Antrag, Aufgabenstellung Semesterarbeit
%
%  Created by Silvan Spross on 2010-10-16.
%
\documentclass[]{scrreprt}
\usepackage[ngerman]{babel}

% Use utf-8 encoding for foreign characters
\usepackage[utf8]{inputenc}

% Setup for fullpage use
\usepackage{fullpage}

% Running Headers and footers
%\usepackage{fancyhdr}

% Multipart figures
%\usepackage{subfigure}

% More symbols
%\usepackage{amsmath}
%\usepackage{amssymb}
%\usepackage{latexsym}

% Surround parts of graphics with box
\usepackage{boxedminipage}

% Package for including code in the document
\usepackage{listings}

% If you want to generate a toc for each chapter (use with book)
\usepackage{minitoc}

% This is now the recommended way for checking for PDFLaTeX:
\usepackage{ifpdf}

\ifpdf
    \usepackage[pdftex]{graphicx}
\else
    \usepackage{graphicx}
\fi

\title{Aufgabenstellung\\
    Semesterarbeit in Informatik}
    
\author{Studierender - Silvan Spross\\
    Projektbetreuer - Beat Seeliger\\
    \\
    HSZ-T - Technische Hochschule Zürich}
    
\date{19. Oktober 2010}

\begin{document}

    \ifpdf
        \DeclareGraphicsExtensions{.pdf, .jpg, .tif}
    \else
        \DeclareGraphicsExtensions{.eps, .jpg}
    \fi

    \maketitle

    \pagenumbering{arabic}

    % \tableofcontents

    \chapter{Aufgabenstellung Semesterarbeit}

    \section{Thema}
    Webapplikation für eine öffentliche Medienbibliothek mit einer 
    API\glossary{name={API}, description={Application Programming Interface}}

    \section{Ausgangslage}
    Viele Personen sind im Besitz von digitalen Medien wie zum Beispiel Filme.
    Im Internet gibt es diverse Plattformen, unter anderem IMDB, um 
    detaillierte Informationen zu diesen Filmen zu finden und sie zu bewerten. 
    
    Die Inhalte werden jedoch von den Plattformbetreibern gepflegt und die 
    Benutzer haben keine Möglichkeit sich à la Wikipedia einzubringen. Auch 
    basieren die Bewertungen immer auf dem Durchschnitt aller Benutzer. Dies
    führt relativ schnell zu einer unnützlichen Bewertung, da die Geschmäcker
    sehr verschieden sind.

    \section{Ziel der Arbeit}
    Wikipedia hat bewiesen, dass öffentliches Pflegen von Inhalten 
    funktioniert. In dieser Arbeit soll ein Prototyp einer Plattform erstellt
    werden, wo Inhalte zu Medien frei erfasst und bearbeitet werden können.
    Als weitere Neuerung sollen Benutzer die Bewertungsanzeige von Medien auf
    frei definierbare Gruppen, wie zum Beispiel ihre Freunde, festlegen 
    können, um so Bewertungen mit gleichem Geschmack zu sehen.
    
    Zusätzlich soll auch der Code der Plattform öffentlich verfügbar sein, 
    damit auch dieser von Dritten weiter gepflegt werden kann. Dazu muss die 
    Wahl der Technologien und Tools wohl überlegt sein, damit sich weitere 
    Entwickler finden lassen.
    
    Folgende Ziele sollen erreicht werden:
    
    \begin{itemize}
        \item Aus der Analyse der Ausgangslage muss hervorgehen, ob eine
            Entwicklung einer solchen Plattform berechtigt ist
        \item Die Plattform muss in der Planungsphase skizziert und
            zwingende und optionale Features müssen aufgelistet werden
        \item Technologien und Tools müssen begründet gewählt werden, damit 
            die Möglichkeit einer Weiterentwicklung durch Dritte hoch ist
        \item Der Prototyp muss auf dem Internet erreichbar und Modelle
            müssen über das Webinterface und eine API bearbeitbar sein
    \end{itemize}
    
    Folgende Punkte werden abgegrenzt, da es den Rahmen der Arbeit sprengen 
    würde:
    
    \begin{itemize}
        \item Die Analysen beschränken sich auf Recherchen im Internet, 
            Büchern und das nähere Umfeld des Studierenden
        \item Es werden keine Umfragen, Erhebungen und Feldstudien 
            durchgeführt
        \item Der Prototyp wird nur über einen Bruchteil der Funktionalität 
            der eigentlichen Plattform verfügen
    \end{itemize}

    \section{Aufgabenstellung}
    Die Ausgangslage soll detaillierter erfasst werden. Welche Plattformen
    gibt es, was bieten sie an und wo liegen ihre Vor- und Nachteile.
    Da ein Schwerpunkt der Arbeit auf der öffentlichen Weiterentwicklung der
    neuen Plattform liegt, müssen die Technologien und Projektunterstützenden 
    Tools evaluiert werden.
    
    Anschliessend soll ein Prototyp mit den gewählten Technologien erstellt
    werden, wo der Ansatz des gemeinsamen Pflegens über das Webinterface und 
    die API demonstriert werden kann.
    
    \begin{itemize}
        \item Analyse der Ausgangslage
        \item Modellierung der Applikation
        \item Evaluation der Technologien
        \item Evaluation der Projektunterstützenden Tools
        \item Entwicklung eines Prototypen mit den gewählten Technologien
        \item Definition von Testfällen
    \end{itemize}

    \section{Erwartete Resultate}
    Die erwarteten Resultate ergeben sich aus der Aufgabenstellung:
    
    \begin{itemize}
        \item Technischer Bericht\footnote{Hochdeutsch, A4, weiss, einseitig 
            bedruckt, zwei gebundene Exemplare und als PDF}:
        \begin{itemize}
            \item Beschreibung der Ausgangslage
            \item Modell der Applikation, Erstellung eines ERMs\footnote{
                Entity-relationship model}
            \item Begründung der gewählten Technologien
            \item Begründung der gewählten Tools
            \item Anmerkungen zur Entwicklung
            \item Resultate der Testfälle
        \end{itemize}
    	\item Lauffähiger Prototyp:
    	\begin{itemize}
            \item Es müssen die Modelle User, Movie und Rating implementiert
                werden
            \item Mehrsprachigkeit umgesetzt (Modell- und statische 
                Inhalte)
            \item API implementiert, worüber man die Modelle Movie und Rating
                bearbeiten kann
        \end{itemize}
    \end{itemize}

    \section{Geplante Termine}
    Die Termine können zum Zeitpunkt des Antrages noch nicht definitiv 
    festgelegt werden. Sofern jedoch die Planung eingehalten werden kann und 
    freie Termine zur Verfügung stehen, sollten die Termine innerhalb der 
    angegebenen Monate liegen.

    \begin{tabbing}
        \hspace*{4cm}\= \kill
    	Kick-Off:               \> 28. Oktober 2010 - 13.00 Uhr \\
    	Review:                 \> 17. November 2010 - 20.00 Uhr \\
    	Schlusspräsentation:    \> 12. Januar 2011 - 13.30 Uhr \\
    \end{tabbing}

    \section{Genehmigung}
    Der Studierende, sein Projektbetreuer und der Studiengangsleiter 
    Informatik erklären sich mit der Aufgabenstellung einverstanden und geben 
    die Arbeit frei zur Erfassung im Einschreibesystem der Hochschule für 
    Technik Zürich.

    \begin{tabbing}
        \hspace*{10cm}\= \kill
    	Silvan Spross, Studierender \> Beat Seeliger, Projektbetreuer \\\\\\
        \line(1,0){150} \> \line(1,0){150} \\\\\\
    	Dr. Olaf Stern, Studiengangsleiter Informatik \\\\\\
        \line(1,0){150}
    \end{tabbing}
    
    \bibliographystyle{plain}
    \bibliography{literaturverzeichnis}
    
\end{document}