%
%  Kickoff Protokoll
%
%  Created by Silvan Spross on 2010-11-17.
%
\documentclass[]{scrreprt}
\usepackage[ngerman]{babel}

% Use utf-8 encoding for foreign characters
\usepackage[utf8]{inputenc}

% Setup for fullpage use
\usepackage{fullpage}

% Running Headers and footers
%\usepackage{fancyhdr}

% Multipart figures
%\usepackage{subfigure}

% More symbols
%\usepackage{amsmath}
%\usepackage{amssymb}
%\usepackage{latexsym}

% Surround parts of graphics with box
\usepackage{boxedminipage}

% Package for including code in the document
\usepackage{listings}

% If you want to generate a toc for each chapter (use with book)
\usepackage{minitoc}

% This is now the recommended way for checking for PDFLaTeX:
\usepackage{ifpdf}

\ifpdf
    \usepackage[pdftex]{graphicx}
\else
    \usepackage{graphicx}
\fi

\title{Design Review Protokoll}
    
\author{Studierender - Silvan Spross\\
    Projektbetreuer - Beat Seeliger\\
    \\
    HSZ-T - Technische Hochschule Zürich}
    
\date{17. November 2010}

\begin{document}

    \ifpdf
        \DeclareGraphicsExtensions{.pdf, .jpg, .tif}
    \else
        \DeclareGraphicsExtensions{.eps, .jpg}
    \fi

    \maketitle

    \pagenumbering{arabic}

    % \tableofcontents

    \chapter{Design Review Protokoll}

    \section{Semesterarbeit}
    Webapplikation für eine öffentliche Medienbibliothek mit einer 
    API\glossary{name={API}, description={Application Programming Interface}}

    \section{Beschlüsse}
    \begin{itemize}
        \item In der Aufgabenstellung wurde das Resultat der zu 
            implementierenden Modelle präzisiert
        \item Der Code des Prototypen wurde einem Review unterzogen und
            grundsätzlich für gut fortgeschritten befunden
        \item Es wurde ein Termin für eine Einführung in UnitTests in Rails
            festgelegt
        \item Das Projekt wurde ein wenig umstrukturiert, damit das
            automatische Deploying mit Capistrano funktioniert
        \item Der Studierende wurde über den weiteren Ablauf der 
            Arbeit informiert
    \end{itemize}
    
\end{document}