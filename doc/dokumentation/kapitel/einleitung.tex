\section{Ausgangslage}
Viele Personen sind im Besitz von digitalen Medien wie zum Beispiel Filme.
Im Internet gibt es diverse Plattformen, um detaillierte Informationen zu 
Filmen zu finden und sie zu bewerten. 

Die Inhalte werden jedoch von den Plattformbetreibern gepflegt und die 
Benutzer haben keine Möglichkeit sich wie bei Wikipedia einzubringen. Auch 
basieren die Bewertungen oft auf dem Durchschnitt aller Benutzer. Dies
führt relativ schnell zu einer unnützlichen Bewertung, da die Geschmäcker
sehr verschieden sind.

\section{Ziel der Arbeit}
Wikipedia hat bewiesen, dass öffentliches Pflegen von Inhalten 
funktioniert. In dieser Arbeit soll ein Prototyp einer Plattform erstellt
werden, auf welcher Inhalte zu Medien frei erfasst und bearbeitet werden können.
Als weitere Neuerung sollen Benutzer die Bewertungsanzeige von Medien auf
frei definierbare Gruppen, wie zum Beispiel ihre Freunde, festlegen 
können, um so Bewertungen mit gleichem Geschmack zu sehen.

Zusätzlich soll auch der Quellcode der Plattform öffentlich verfügbar sein, 
damit auch dieser von Dritten weiter gepflegt werden kann. Dazu muss die 
Wahl der Technologien und Tools wohl überlegt sein, damit sich weitere 
Entwickler finden lassen.

\section{Abgrenzung}
Die Analysen beschränken sich auf Recherchen im Internet, Büchern und das 
nähere Umfeld des Studierenden. Es werden keine Umfragen, Erhebungen und 
Feldstudien durchgeführt. Der Prototyp wird nur über einen Bruchteil der 
Funktionalität der eigentlichen Plattform verfügen, da es den Rahmen
dieser Arbeit sprengen würde.

Die Medien wurden auf das Medium Film beschränkt, damit für eine 
Kategorie möglichst genaue Aussagen getroffen werden können.

\section{Richtlinien}
Folgende Dokumente mit Richtlinien der Hochschule für Technik Zürich 
wurden für die Semesterarbeit berücksichtigt:

\begin{itemize}
    \item Reglement \cite{hsz_reglement}
    \item Ablauf \cite{hsz_ablauf}
    \item Bewertungskriterien \cite{hsz_bewertungskriterien}
\end{itemize}

\section{Sprache}
Die Semesterarbeit wurde in deutscher Sprache verfasst. Englische Ausdrücke 
wurden immer dort verwendet, wo diese im Sprachgebrauch in den verwendeten 
Programmen genau so gebraucht werden.

Aus Gründen der besseren Lesbarkeit der Semesterarbeit wurde teilweise auf 
die Nennung beider Geschlechter verzichtet. In diesen Fällen ist die 
weibliche Form ausdrücklich inbegriffen.