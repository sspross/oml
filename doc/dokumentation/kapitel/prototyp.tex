\section{Entwicklungsumgebung}
Da nun die zu verwendenden Technologien und Tools gewählt sind, kann mit der
eigentlichen Entwicklung des Prototypen begonnen werden. Als erster Schritt
muss die Entwicklungsumgebung gewählt und eingerichtet werden

MacOS X, Textmate, Terminal

\subsection{Versionsverwaltungssystem}
git develop, master

\section{Umsetzung}

um project zero zum laufen zu bringen war nötig:\\
rmv installieren, nokogiri, bundle install

\section{Testen}

\subsection{Testfälle}

\subsection{Testresultate}

\section{Deployment}
Systeme: local, staging, production\\
Datenbanken: sqlite, mysql