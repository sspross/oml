\section{Angewandte Methoden}
Folgende Methoden wurden in dieser Arbeit zur Unterstützung verwendet. 

\subsection{Nutzwertanalyse}
Die Nutzwertanalyse gehört zu den quantitativen nicht-monetären Analysemethoden 
der Entscheidungstheorie. In Form einer oder mehreren 
Entscheidungsmatrizen werden Kriterien und Argumente, welche letztendlich eine 
Entscheidung bestimmen, einer genauen Prüfung unterzogen und auf eine Auswahl 
angewandt \cite{nutzwertanalyse}.

\subsection{Wireframing}
Alternativ zum Ausdruck ``Mock-up'' wird der Begriff ``Wireframe'' benutzt, um einen 
sehr frühen konzeptuellen Prototypen einer Website oder eines Software-Frontends 
darzustellen. Bezogen auf eine Website sollten Elemente wie Navigation und 
Inhaltsbereiche Teil dieses Skeletts sein \cite{wireframe}.

\subsection{Entity Relationship Model}
Das ERM (``Entity-Relationship-Modell'') dient dazu, im Rahmen der semantischen 
Datenmodellierung einen Ausschnitt der realen Welt zu beschreiben. 
Das ER-Modell besteht jeweils aus einer Grafik und einer Erläuterung \cite{erm}.

\section{Weiterentwicklung}
Der Prototyp befindet sich zum Zeitpunkt der Abgabe dieser Arbeit in einem
lauffähigen Zustand und kann unter \url{http://oml.orwell.ch} aufgerufen werden. 
Das automatische Deployment ist so eingerichtet, dass immer der aktuelle `master' 
Branch, der auf ``GitHub'' verfügbar ist, auf dem Server installiert wird.

Hinzukommende Drittentwickler können das Projekt auf ``GitHub'' ansehen und
zu sich `clonen' \cite{clone} und `forken' \cite{fork}. Das bedeutet das es
Drittentwicklern sehr einfach gemacht ist, sich in das Projekt einzubringen
und mitzuarbeiten.

Wenn jemand das Projekt lokal auf seinem Rechner einrichten möchte, kann er
folgende Kommandos in einer Konsole ausführen\footnote{Vorausgesetzt auf seinem
Rechner ist `Ruby' Version 1.8.7 oder höher, `RubyGems' Version 1.3.7 oder 
höher und `Git' Version 1.7.1.1 oder höher installiert.}:

\begin{verbatim}
    git clone git@github.com:sspross/oml.git
    cd oml/
    bundle install --without=production
    cp config/database.yml.sqlite3 config/database.yml
    rake db:migrate
\end{verbatim}

Danach ist das Projekt auf den lokalen Rechner kopiert und alle Abhängigkeiten
sind installiert. Nun können die Tests mit folgendem Befehl ausgeführt werden:

\begin{verbatim}
    rake test
\end{verbatim}

Der Server kann lokal mit folgendem Befehl gestartet werden:

\begin{verbatim}
    rails server
\end{verbatim}

Nun ist eine lauffähige, lokale Instanz der Webapplikation unter 
\url{http://localhost:3000/} erreichbar.  

\section{Erfahrungsbericht}
Die Tests zu implementieren hat Spass gemacht, da sie sehr einfach zu schreiben
waren.