\section{Konkurrenz}
Es existieren diverse Plattformen die Informationen zu Filmen zur Verfügung stellen.
Die laut Google \cite{movie_informations} Bekannteste unter ihnen ist ``The Internet Movie Database'',
besser bekannt unter der Abkürzung ``IMDB''. Sie bietet detaillierte Informationen
zu existierenden Filmen, solche die zur Zeit gedreht werden und sogar über jene, die
sich erst in der Planungsphase befinden.

Es existieren natürlich noch viele weitere Plattformen und die Grössten unter 
ihnen möchte ich nach den folgenden drei Kriterien beurteilen:

\begin{table}[h]
\begin{center}
    \begin{tabular}{lp{12cm}l}
        \toprule Nr & Beschreibung \\
        \midrule 1 & Der Besucher kann die Beschreibung, welche den Inhalt 
                     des Films in einer kurzen Zusammenfassung wiedergibt, mitbestimmen. \\
        \midrule 2 & Der Besucher kann die Filme bewerten. \\
        \midrule 3 & Der Besucher hat die Möglichkeit für jeden Film eine zusätzliche
                     Bewertung zu sehen, die nur über ihn und seine Freunde berechnet
                     wird und somit für seinen Geschmack aussagekräftiger ist. \\
        \bottomrule
    \end{tabular}
    \caption{Meine Bewertungskriterien einer guten Filmbewertungsplattform}
    \label{tab:bewertungskriterien}
\end{center}
\end{table}

Der Prototyp namens ``OpenMediaLibrary'', kurz ``OML'', der in dieser Arbeit erstellt werden soll, 
wird alle drei Kriterien erfüllen, da diese meines Erachtens die Grundlage für eine gute 
öffentliche Filmdatenbank sind.

In der Tabelle \ref{tab:plattformen} habe ich die bekanntesten 
Plattformen untersucht und nach den definierten Kriterien beurteilt. Ein `x' bedeutet,
dass die Plattform das Kriterium erfüllt und ein `-', dass sie es nicht erfüllt.

\begin{table}[h]
\begin{center}
    \begin{tabular}{lllccc}
        \toprule Nr & Name & URL & \multicolumn{3}{c}{Kriterien} \\ & & & 1 & 2 & 3 \\
        \midrule 1 & The Internet Movie Database & imdb.com & - & x & - \\
        \midrule 2 & Deutsche Film- und Medienbewertung & fbw-filmbewertung.com & - & - & - \\
        \midrule 3 & Online-Filmdatebank & ofdb.de & - & x & - \\
        \midrule 4 & Wikipedia & wikipedia.com & x & - & - \\
        \midrule 5 & Rotten Tomatoes & rottentomatoes.com & - & x & - \\
        \bottomrule
    \end{tabular}
    \caption{Die bekanntesten existierenden Plattformen}
    \label{tab:plattformen}
\end{center}
\end{table}

Wie in der Ausgangslage schon angenommen hat die Recherche ergeben, dass nur
wenige Plattformen die Mitarbeit der Besucher erlaubt. Fast alle bieten
eine Bewertung an, jedoch keine eine Bewertung, die sich nur auf den
Freundeskreis des Betrachters bezieht.

\section{Funktionsumfang}
Relevant sind jedoch nicht nur meine Bewertungskriterien aus der Tabelle \ref{tab:bewertungskriterien},
sondern auch der allgemeine Funktionsumfang einer Filmbewertungsplattform.
Deshalb bewerte ich die Plattformen inklusive der geplanten neuen Plattform ``OML'' 
anhand von mir ausgewählten Funktionen, die ich dann mit einer Gewichtung,
im Hinblick auf was meines Erachtens eine gute und öffentliche Plattform auszeichnet, versehe.

Folgende sechs Funktionen bilden aus meiner Sicht die Basis einer solchen Plattform: 

\begin{enumerate}
    \item \textbf{Filmbeschreibung}\\
          Ein Film hat eine Beschreibung, welche ohne jegliche Kritik den Inhalt
          des Filmes in einer kurzen Zusammenfassung wiedergibt. Bei dieser Funktion
          wird jedoch nicht darauf geachtet, ob der Benutzer darauf Einfluss nehmen
          kann oder nicht.
    \item \textbf{Redaktionelle Bewertung}\\
          Ein Film hat eine Bewertung, die von den Verfassern der Filmbeschreibung 
          abgegeben wird. Also keine berechnete, sondern eine subjektive Bewertung
          des oder der Autoren der Filmbeschreibung.
    \item \textbf{Bewertung durch Benutzer}\\
          Ein Film kann durch die Benutzer, die sich auf der Plattform
          registrieren müssen, bewertet werden. Die Bewertungen aller Benutzer
          fliessen dann in eine Gesamtbewertung. Die Gesamtbewertung wird berechnet,
          indem man den Mittelwert der Summe aller Bewertungen nimmt.
    \item \textbf{Kommentare von Benutzern}\\
          Benutzer können individuelle Kommentare zu einem Film abgeben. Diese
          können ergänzend zu einer Bewertung, aber auch ohne, hinzugefügt werden.
    \item \textbf{Bewertung des Freundeskreises}\\
          Ein Benutzer kann die Bewertung eines Filmes im Kreise seiner Freunde
          betrachten und so eine, für ihn, aussagekräftigere Bewertung erhalten.
          Diese wird berechnet, indem man den Mittelwert der Summe aller Bewertungen
          seiner Freunde und seiner eigenen nimmt.
    \item \textbf{Weiterführende Informationen}\\
          Es können weitere Verweise und Hinweise, die nützliche Informationen
          zu einem Film bieten, abgebildet werden. Dies kann in Form von zusätzlichen
          Quellenangaben, Kaufinformationen oder rein informativem Texten sein.
\end{enumerate}

\subsection{Gewichtung}
Nun versehe ich die genannten Funktionen mit einer Gewichtung. Diese spiegelt die
Relevanz der Funktion wieder. Die Gewichtung ist von mir vergeben, jedoch auch in Bezug
auf Diskussionen in meinem näheren Umfeld.

In der Tabelle \ref{tab:funktionen} habe ich die Funktionen und deren Gewichtung
aufgelistet. Die Gewichtung geht von 1 bis 5, wobei 1 für `unwichtig' und 5 für
`sehr wichtig' steht. Es dürfen nur ganze Zahlen verwendet werden, da ich 
absichtlich eine Gewichtung ohne exakte Mitte gewählt habe. Dies dient der
Aussagekraft und der besseren Abgrenzbarkeit.

\begin{table}[h]
\begin{center}
    \begin{tabular}{llc}
        \toprule Nr & Funktion & Gewichtung \\
        \midrule 1 & Filmbeschreibung & 5 \\
        \midrule 2 & Redaktionelle Bewertung & 2 \\
        \midrule 3 & Bewertung durch Benutzer & 4 \\
        \midrule 4 & Kommentare von Benutzern & 1 \\
        \midrule 5 & Bewertung des Freundeskreises & 4 \\
        \midrule 6 & Weiterführende Informationen & 3 \\
        \bottomrule
    \end{tabular}
    \caption{Funktionen einer guten Filmbewertungsplattform}
    \label{tab:funktionen}
\end{center}
\end{table}

\subsection{Direkter Vergleich}
Nun übertrage ich die Funktionen inklusive Gewichtung auf die einzelnen Plattformen
um die bestehenden mit der geplanten Plattform vergleichen zu können.

Als vergleichbare Grundlage habe ich bei den bestehenden Plattformen den Film 
``Das Experiment'' aus dem Jahre 2001 gewählt. In der Tabelle \ref{tab:deeplinks} 
sind die genauen Webadressen zu dem Film auf den einzelnen Plattformen aufgelistet. 
Die Deeplinks sind zum Zeitpunkt der Erstellung dieser Arbeit gültige Adressen.

\begin{table}[h]
\begin{center}
    \begin{tabular}{lll}
        \toprule Plattform & Deeplink \\
        \midrule 1 & \url{http://www.imdb.com/title/tt0250258/} \\
        \midrule 2 & \url{http://www.fbw-filmbewertung.com/film/das_experiment_1} \\
        \midrule 3 & \url{http://www.ofdb.de/film/3681,Das-Experiment} \\
        \midrule 4 & \url{http://de.wikipedia.org/wiki/Das_Experiment_%28Film%29} \\
        \midrule 5 & \url{http://www.rottentomatoes.com/m/1116582-experiment/} \\
        \bottomrule
    \end{tabular}
    \caption{Deeplinks zu ``Das Experiment'' der existierenden Plattformen}
    \label{tab:deeplinks}
\end{center}
\end{table}

Wenn eine Plattform die gewünschte Funktion unterstützt, erhält sie die volle Gewichtung,
wenn nicht erhält sie den Wert `0'. Somit wird der Erfüllungsgrad der einzelnen Plattformen
immer als `erfüllt' oder `nicht erfüllt' angesehen. Zur bessere Darstellung wird der Wert `0' 
mit einem `-' eingetragen.

Die Bewertung der eigenen geplanten Plattform Nr. 6 ``OpenMediaLibrary'', basiert auf
der Annahme, dass alle Funktionen umgesetzt werden können, ausser die Funktionalität Nr. 6 
``Weiterführende Informationen''. Auf diese habe ich ganz verzichtet, da sie zu Beginn 
nicht zwingend notwendig ist und damit nicht die volle Punktzahl von 19 erreicht wird.

In der nachstehenden Tabelle \ref{tab:funktionen_vergleich} sind die Plattformen und
die Funktionen in einer Matrix aufgelistet und mit den Gewichtungen versehen. Als
Resultat dient die Summe der Gewichtungen pro Plattform. Um so höher die Summe, umso
mehr relevante Funktionen sind in der Plattform abgebildet.

\begin{table}[h]
\begin{center}
    \begin{tabular}{llccccccc}
        \toprule
        \multicolumn{2}{c}{Plattform} & \multicolumn{6}{c}{Funktion} & \textbf{Summe} \\
        \multicolumn{2}{c}{} & 1 & 2 & 3 & 4 & 5 & 6 & \\ 
        \midrule 1 & The Internet Movie Database & 5 & - & 4 & 1 & - & 3 & \textbf{13} \\
        \midrule 2 & Deutsche Film- und Medienbewertung & 5 & 2 & - & - & - & - & \textbf{7} \\ 
        \midrule 3 & Online-Filmdatebank & 5 & - & 4 & - & - & 3 & \textbf{12} \\ 
        \midrule 4 & Wikipedia & 5 & 2 & - & - & - & 3 & \textbf{10} \\ 
        \midrule 5 & Rotten Tomatoes & 5 & 2 & 4 & 1 & - & - & \textbf{12} \\ 
        \midrule 6 & OpenMediaLibrary & 5 & 2 & 4 & 1 & 4 & - & \textbf{16} \\ 
        \bottomrule
    \end{tabular}
    \caption{Funktionsvergleich der Plattformen}
    \label{tab:funktionen_vergleich}
\end{center}
\end{table}

\section{Zusammenfassung}
Aus den einzelnen Summen in der Tabelle \ref{tab:funktionen_vergleich} geht hervor,
dass es sich lohnt, den Prototypen der neuen Plattform namens ``OpenMediaLibrary'' zu erstellen. 
Ich stütze mich hierbei, wie erwähnt, vor allem auf mein näheres Umfeld, wo diese Funktionalität 
ebenso gewünscht und geschätzt würde.

Falls dieses Projekt über den Prototyp-Status weiterverfolgt werden sollte, lohnt
es sich eine breitere, wenn möglich, weltweite Umfrage über den Funktionsumfang
durchzuführen.