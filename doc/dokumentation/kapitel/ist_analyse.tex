\section{Konkurrenz}
Es existieren diverse Plattformen die Informationen zu Filmen zur Verfügung stellen.
Die Bekannteste \cite{movie_informations} unter ihnen ist ``The Internet Movie Database'',
besser bekannt unter der Abkürzung ``IMDB''. Sie bietet detaillierte Informationen
zu existierenden Filmen, solche die zur Zeit gedreht werden und sogar über jene, die
sich erst in der Planung befinden.

Es existieren natürlich noch viele weitere Plattformen und die Grössten darunter
möchte ich unter den folgenden drei Kriterien beurteilen:

\begin{itemize}
    \item I: Der Besucher kann auf den beschreibenden Inhalt Einfluss nehmen.
    \item R: Der Besucher kann die Filme bewerten. 
    \item FR: Der Besucher kann eine eigene Bewertung unter seinen Freunden führen.
\end{itemize}

Der Prototyp, der in dieser Arbeit erstellt werden soll, soll alle drei Kriterien
erfüllen, da diese meines Erachtens die Grundlage für eine gute öffentliche Filmdatenbank 
sind.

In der Tabelle \ref{tab:plattformen} habe ich die bekannteren und unbekannteren 
Plattformen untersucht und unter den definierten Kriterien betrachtet:

\begin{table}[h]
\begin{center}
    \begin{tabular}{lllccc}
        \toprule Nr & Name & URL & I & R & FR \\
        \midrule 1 & The Internet Movie Database & imdb.com & - & x & - \\
        \midrule 2 & Deutsche Film- und Medienbewertung & fbw-filmbewertung.com & - & - & - \\
        \midrule 3 & Online-Filmdatebank & ofdb.de & - & x & - \\
        \midrule 4 & Wikipedia & wikipedia.com & x & - & - \\
        \midrule 5 & Rotten Tomatoes & rottentomatoes.com & - & x & - \\
        \bottomrule
    \end{tabular}
    \caption{Existierende Plattformen}
    \label{tab:plattformen}
\end{center}
\end{table}

Wie in der Ausgangslage schon angenommen hat die Recherche ergeben, dass nur
wenige Plattformen die Mitarbeit der Besucher erlaubt. Fast alle bieten
eine Bewertung an, jedoch keine eine Bewertung die sich nur auf den
Freundeskreis des Betrachters bezieht.

\section{Funktionsumfang}
Nun bewerte ich die Plattformen inklusive der geplanten neuen Plattform anhand
ausgewählten Funktionen mit einer Gewichtung im Hinblick auf was eine gute und
öffentliche Plattform auszeichnet.
Folgende Funktionen werden gewichtet:

\begin{enumerate}
    \item \textbf{Kritikfreie Filmbeschreibung}\\
          Ein Film hat eine Beschreibung, welche ohne jegliche Kritik den Inhalt
          des Filmes in einer kurzen Zusammenfassung wiedergibt.
    \item \textbf{Redaktionelle Bewertung}\\
          Ein Film hat eine Bewertung, die von den Verfassern der Filmbeschreibung 
          abgegeben wird.
    \item \textbf{Bewertung durch Benutzer}\\
          Ein Film kann durch die Benutzer, die sich auf der Plattform
          registrieren müssen, bewertet werden.
    \item \textbf{Kommentare von Benutzern}\\
          Benutzer können individuelle Kommentare zu einen Film abgeben. Diese
          können ergänzend zu einer Bewertung, aber auch ohne, hinzugefügt werden.
    \item \textbf{Bewertung des Freundeskreises}\\
          Ein Benutzer kann die Bewertung eines Filmes im Kreise seiner Freunde
          betrachten und so eine, für ihn, aussagekräftigere Bewertung erhalten.
    \item \textbf{Weiterführende Informationen}\\
          Es können weitere Verweise und Hinweise, die nützliche Informationen
          zu einen Film bieten, abgebildet werden.
\end{enumerate}

\subsection{Gewichtung}
In der Tabelle \ref{tab:funktionen} habe ich die Funktionen und deren Gewichtung
aufgelistet. Die Gewichtung geht von 1 bis 5, wobei 1 für `unwichtig' und 5 für
`sehr wichtig' steht. Es dürfen nur ganze Zahlen verwendet werden, da ich 
absichtlich eine Gewichtung ohne exakte Mitte gewählt habe. Dies dient der
Aussagekraft und der besseren Abgrenzbarkeit.

\begin{table}[h]
\begin{center}
    \begin{tabular}{llc}
        \toprule Nr & Funktion & Gewichtung \\
        \midrule 1 & Kritikfreie Filmbeschreibung & 5 \\
        \midrule 2 & Redaktionelle Bewertung & 2 \\
        \midrule 3 & Bewertung durch Benutzer & 4 \\
        \midrule 4 & Kommentare von Benutzern & 1 \\
        \midrule 5 & Bewertung des Freundeskreises & 5 \\
        \midrule 6 & Weiterführende Informationen & 3 \\
        \bottomrule
    \end{tabular}
    \caption{Funktionen einer guten Filmbewertungsplattform}
    \label{tab:funktionen}
\end{center}
\end{table}

\subsection{Direkter Vergleich}
Nun übertrage ich die Funktionen inklusive Gewichtung auf die einzelnen Plattformen
um die bestehenden mit der geplanten Plattform vergleichen zu können.

Als vergleichbare Grundlage habe ich bei den bestehenden Plattformen den Film 
``Das Experiment'' aus dem Jahre 2001 gewählt. In der Tabelle \ref{tab:deeplinks} 
sind die genauen Webadressen zu dem Film auf den einzelnen Plattformen aufgelistet. 
Die Deeplinks sind zum Zeitpunkt der Erstellung dieser Arbeit gültige Adressen.

\begin{table}[h]
\begin{center}
    \begin{tabular}{lll}
        \toprule Plattform & Deeplink \\
        \midrule 1 & \url{http://www.imdb.com/title/tt0250258/} \\
        \midrule 2 & \url{http://www.fbw-filmbewertung.com/film/das_experiment_1} \\
        \midrule 3 & \url{http://www.ofdb.de/film/3681,Das-Experiment} \\
        \midrule 4 & \url{http://de.wikipedia.org/wiki/Das_Experiment_%28Film%29} \\
        \midrule 5 & \url{http://www.rottentomatoes.com/m/1116582-experiment/} \\
        \bottomrule
    \end{tabular}
    \caption{Deeplinks zu ``Das Experiment'' der existierenden Plattformen}
    \label{tab:deeplinks}
\end{center}
\end{table}

Wenn eine Plattform die gewünschte Funktion unterstützt, erhält sie die volle Gewichtung,
wenn nicht erhält sie den Wert `0'. Somit wird der Erfüllungsgrad der einzelnen Plattformen
immer als `erfüllt' oder `nicht erfüllt' angesehen. Für eine bessere Darstellung wird der Wert `0'
mit einem `-' eingetragen. 

\begin{table}[h]
\begin{center}
    \begin{tabular}{llccccccc}
        \toprule
        \multicolumn{2}{c}{Plattform} & \multicolumn{6}{c}{Funktion} & \textbf{Summe} \\
        \multicolumn{2}{c}{} & 1 & 2 & 3 & 4 & 5 & 6 & \\ 
        \midrule 1 & The Internet Movie Database & 5 & - & 4 & 1 & - & 3 & \textbf{13} \\
        \midrule 2 & Deutsche Film- und Medienbewertung & 5 & 2 & - & - & - & - & \textbf{7} \\ 
        \midrule 3 & Online-Filmdatebank & 5 & - & 4 & - & - & 3 & \textbf{12} \\ 
        \midrule 4 & Wikipedia & 5 & 2 & - & - & - & 3 & \textbf{10} \\ 
        \midrule 5 & Rotten Tomatoes & 5 & 2 & 4 & 1 & - & - & \textbf{12} \\ 
        \midrule 6 & OpenMediaLibrary & 5 & 2 & 4 & 1 & 5 & - & \textbf{17} \\ 
        \bottomrule
    \end{tabular}
    \caption{Funktionsvergleich der Plattformen}
    \label{tab:funktionen_vergleich}
\end{center}
\end{table}

\section{Konklusion}
Aus den einzelnen Summen in der Tabelle \ref{tab:funktionen_vergleich} geht hervor,
dass es sich lohnt einen Prototypen der neuen Plattform zu erstellen. Ich stütze
mich hierbei auch auf mein näheres Umfeld, wo diese Funktionalität ebenso gewünscht 
und geschätzt wird.

Falls dieses Projekt über den Prototyp-Status weiterverfolgt werden sollte, lohnt
es sich eine breitere, wenn möglich, weltweite Umfrage über den Funktionsumfang
durchzuführen.