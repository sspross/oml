\section{Konkurrenz}
Es existieren diverse Plattformen die Informationen zu Filmen

\begin{itemize}
    \item I: Der Besucher kann auf den Beschreibenden Inhalt Einfluss nehmen.
    \item R: Der Besucher kann die Filme bewerten. 
    \item FR: Der Besucher kann eine eigene Bewertung unter seinen Freunden führen.
\end{itemize}

In der Tabelle \ref{tab:plattformen} habe ich diverse bekannte und 
unbekanntere Plattformen untersucht und unter den definierten Merkmalen betrachtet:

\begin{table}[h]
\begin{center}
    \begin{tabular}{lllccc}
        \toprule Nr & Adresse & URL & I & R & FR \\
        \midrule 1 & Deutsche Film- und Medienbewertung & fbw-filmbewertung.com & - & - & - \\
        \midrule 2 & Online-Filmdatebank & ofdb.de & - & x & - \\
        \midrule 3 & Wikipedia & wikipedia.com & x & - & - \\
        \midrule 4 & The Internet Movie Database & imdb.com & - & x & - \\
        \bottomrule
    \end{tabular}
    \caption{Existierende Plattformen}
    \label{tab:plattformen}
\end{center}
\end{table}

Wie in der Ausgangslage schon angenommen hat die Recherche ergeben, dass nur
wenige Plattformen die Mitarbeit der Besucher erlaubt. Fast alle bieten
eine Bewertung an, jedoch keine eine Bewertung die sich nur auf den
Freundeskreis des Betrachters bezieht.

\section{Funktionsumfang}
Übersicht über die Plattformen und ihre Funktionen mit Gewichtung.

\section{Konklusion}
Die Schlüsse die aus der Ist Analyse gezogen werden.