\subsubsection{Benutzer}
Ein vollständig abgefüllter Benutzer muss erfolgreich validiert werden können,
wenn man ihn speichert:

\begin{verbatim}
    context "Minimal User not saved" do
      setup { @user = Factory.build(:user) }
      should("validate") { assert @user.valid? }
    end
\end{verbatim}

Da die E-Mailadresse des Benutzers gleichzeitig seinen Benutzernamen darstellt,
muss sichergestellt werden, dass jede E-Mailadresse nur einmal verwendet werden
kann:

\begin{verbatim}
    context "Minimal user" do
      setup { @user = Factory(:user) }
      should validate_uniqueness_of(:email)
    end
\end{verbatim}

Wenn ein neuer Benutzer erstellt wird, dürfen noch keine Freundschaftsbeziehungen
bestehen und keine Bewertungen existieren:

\begin{verbatim}
    context "New user" do
      setup { @user = Factory(:user) }

      should "have no friends" do
        assert_equal @user.friends.size, 0
        assert_equal @user.inverse_friendships.size, 0
        assert_equal @user.direct_friends.size, 0
        assert_equal @user.inverse_friends.size, 0
        assert_equal @user.pending_friends.size, 0
        assert_equal @user.requested_friendships.size, 0
      end

      should "have no ratings" do
        assert_equal @user.ratings.size, 0
      end
    end
\end{verbatim}

\subsubsection{Bewertung}
Eine Bewertung muss einen gültigen Wert zwischen 1 und 10 enthalten:

\begin{verbatim}
    should "need a integer value between 1 and 10" do
      @rating.value = nil
      assert_equal @rating.valid?, false
      @rating.value = "adsf"
      assert_equal @rating.valid?, false
      @rating.value = 0
      assert_equal @rating.valid?, false
      @rating.value = 11
      assert_equal @rating.valid?, false
      @rating.value = 5
      assert_equal @rating.valid?, true
    end
\end{verbatim}

Ein Benutzer hat einen Film solange nicht bewertet, bis er ausdrücklich
eine Bewertung abgibt:

\begin{verbatim}
    should "be added to a movie and user" do
      assert_does_not_contain @silvan.ratings, @rating
      assert_does_not_contain @movie1.ratings, @rating
      assert_equal @movie1.ratings.size, 0
      assert_equal @silvan.ratings.size, 0
      assert_equal @rating.movie, nil
      assert_equal @rating.user, nil
      
      @rating.value = 5
      @rating.movie = @movie1
      @rating.user = @silvan
      @rating.save!
      
      @movie1.reload
      @silvan.reload
      
      assert_contains @silvan.ratings, @rating
      assert_contains @movie1.ratings, @rating
      assert_equal @movie1.ratings.size, 1
      assert_equal @silvan.ratings.size, 1
      assert_equal @rating.movie, @movie1
      assert_equal @rating.user, @silvan
    end
\end{verbatim}

Die Benutzer sehen verschiedene Bewertungen ihres Freundeskreises, da sie
verschiedene Freundschaften eingegangen sind:

\begin{verbatim}
    setup do 
      @silvan = Factory(:user)
      @stefan = Factory(:user)
      @friendship = Factory(:friendship, :user => @silvan, 
                            :friend => @stefan, :approved => true)
      @marcos = Factory(:user)
      @friendship = Factory(:friendship, :user => @marcos, 
                            :friend => @silvan, :approved => true)
      @muster = Factory(:user)
      @movie1 = Factory(:movie)
    end
    
    should "have different averages" do
      assert_equal @movie1.ratings.size, 0
      assert_equal @movie1.rating_all, nil
      assert_equal @movie1.rating_friends_of(@silvan), nil 
      assert_equal @movie1.rating_friends_of(@stefan), nil 
      assert_equal @movie1.rating_friends_of(@muster), nil 
      
      @movie1.ratings << Factory(:rating, :user => @silvan, :value => 4)
      assert_equal @movie1.ratings.size, 1
      assert_equal @movie1.rating_all, 4
      assert_equal @movie1.rating_friends_of(@silvan), nil 
      assert_equal @movie1.rating_friends_of(@stefan), 4 
      assert_equal @movie1.rating_friends_of(@muster), nil
      
      @movie1.ratings << Factory(:rating, :user => @muster, :value => 6)
      assert_equal @movie1.ratings.size, 2
      assert_equal @movie1.rating_all, 5
      assert_equal @movie1.rating_friends_of(@silvan), nil 
      assert_equal @movie1.rating_friends_of(@stefan), 4 
      assert_equal @movie1.rating_friends_of(@muster), nil
      
      @movie1.ratings << Factory(:rating, :user => @stefan, :value => 8)
      assert_equal @movie1.ratings.size, 3
      assert_equal @movie1.rating_all, 6
      assert_equal @movie1.rating_friends_of(@silvan), 8 
      assert_equal @movie1.rating_friends_of(@stefan), 4 
      assert_equal @movie1.rating_friends_of(@muster), nil
      
      @movie1.ratings << Factory(:rating, :user => @marcos, :value => 8)
      assert_equal @movie1.ratings.size, 4
      assert_equal @movie1.rating_all, 6.5
      assert_equal @movie1.rating_friends_of(@silvan), 8
      assert_equal @movie1.rating_friends_of(@stefan), 4 
      assert_equal @movie1.rating_friends_of(@muster), nil
      assert_equal @movie1.rating_friends_of(@marcos), 4
    end
\end{verbatim}

\subsubsection{Film}
Ein neuer Film hat zu Beginn noch keine Bewertungen:

\begin{verbatim}
    should "have not any ratings first" do
      assert_does_not_contain @movie.ratings, @rating
      assert_equal @movie.ratings.size, 0
      assert_contains @user.ratings, @rating
    end
\end{verbatim}

Sobald ein Film bewertet wurde, kann kann ein Mittelwert der Bewertungen
berechnet werden:

\begin{verbatim}
    setup do
       @movie = Factory(:movie)
       @user = Factory(:user)
       @rating = Factory(:rating, :user => @user, :value => 5)
    end
    
    should "can get rated" do
      @movie.ratings << @rating
      @movie.save!
      
      assert_contains @movie.ratings, @rating
      assert_equal @movie.ratings.size, 1
      assert_equal @rating.movie, @movie
      assert_equal @movie.rating_all, 5
      assert_equal @movie.rating_friends_of(@user), nil
    end
\end{verbatim}

Wenn ein Film von der Plattform entfernt wird, werden auch die Bewertungen
der Benutzer entfernt:

\begin{verbatim}
    should "also destroy his ratings" do
      assert_contains Rating.all, @rating
      
      @movie.ratings << @rating
      @movie.save!
      assert_contains @movie.ratings, @rating
      
      @movie.destroy
      assert_does_not_contain Rating.all, @rating
    end
\end{verbatim}

\subsubsection{Freundschaft}
Zu Beginn sind zwei Personen nicht befreundet:

\begin{verbatim}
    should "have no friendship first" do
      assert_does_not_contain @stefan.friends, @silvan
      assert_does_not_contain @silvan.friends, @stefan
    end
\end{verbatim}

Eine Freundschaft kann von einem Benutzer beantragt werden:

\begin{verbatim}
    setup do
      @stefan = Factory(:user)
      @silvan = Factory(:user)
      @friendship = Factory(:friendship, :user => @silvan, :friend => @stefan)
    end
    
    should "can request a friendship" do
      assert_contains @silvan.pending_friends, @stefan
      assert_contains @stefan.requesting_friends, @silvan
    end
\end{verbatim}

Die Freundschaftsanfrage kann akzeptiert werden:

\begin{verbatim}
    should "can accept a friendship request" do
      request = @stefan.requested_friendships.first
      request.approved = true
      request.save!
      
      assert_contains @stefan.friends, @silvan
      assert_contains @silvan.friends, @stefan
      
      assert_does_not_contain @silvan.pending_friends, @stefan
      assert_does_not_contain @stefan.requesting_friends, @silvan
    end
\end{verbatim}

Die Freundschaft kann aber auch wieder aufgelöst werden:

\begin{verbatim}
    should "can end up with a friendship" do
      friendship = @stefan.inverse_friendships.first
      friendship.approved = false
      friendship.save!
      
      assert_does_not_contain @stefan.friends, @silvan
      assert_does_not_contain @silvan.friends, @stefan
      
      assert_contains @silvan.pending_friends, @stefan
      assert_contains @stefan.requesting_friends, @silvan 
    end
\end{verbatim}