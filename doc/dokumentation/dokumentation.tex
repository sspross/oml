%
%  Semesterarbeit Dokumentation
%
%  Created by Silvan Spross on 2010-10-21.
%
\documentclass[abstracton,liststotoc,bibtotoc]{scrreprt}
\usepackage[ngerman]{babel}

% Use utf-8 encoding for foreign characters
\usepackage[utf8]{inputenc}

% Setup for fullpage use
\usepackage{fullpage}

% Running Headers and footers
%\usepackage{fancyhdr}

% Multipart figures
%\usepackage{subfigure}

% More symbols
%\usepackage{amsmath}
%\usepackage{amssymb}
%\usepackage{latexsym}

% Surround parts of graphics with box
\usepackage{boxedminipage}

% Package for including code in the document
\usepackage{listings}

% If you want to generate a toc for each chapter (use with book)
\usepackage{minitoc}

% This is now the recommended way for checking for PDFLaTeX:
\usepackage{ifpdf}

\ifpdf
    \usepackage[pdftex]{graphicx}
\else
    \usepackage{graphicx}
\fi

\title{Webapplikation für eine öffentliche Medienbibliothek mit einer API}

\author{Studierender - Silvan Spross\\
    Projektbetreuer - Beat Seeliger\\
    \\
    HSZ-T - Technische Hochschule Zürich}
    
\date{Oktober 2010 bis Februar 2011}

\begin{document}

    \ifpdf
        \DeclareGraphicsExtensions{.pdf, .jpg, .tif}
    \else
        \DeclareGraphicsExtensions{.eps, .jpg}
    \fi

    \maketitle
    
    \pagenumbering{Roman}
    
    \begin{abstract}
        
    \end{abstract}

    \tableofcontents
    
    \pagenumbering{arabic}

    \chapter{Einleitung}
    \section{Ausgangslage}
    Viele Personen sind im Besitz von digitalen Medien wie zum Beispiel Filme.
    Im Internet gibt es diverse Plattformen, unter anderem IMDB, um 
    detaillierte Informationen zu diesen Filmen zu finden und sie zu bewerten. 

    Die Inhalte werden jedoch von den Plattformbetreibern gepflegt und die 
    Benutzer haben keine Möglichkeit sich à la Wikipedia einzubringen. Auch 
    basieren die Bewertungen immer auf dem Durchschnitt aller Benutzer. Dies
    führt relativ schnell zu einer unnützlichen Bewertung, da die Geschmäcker
    sehr verschieden sind.
    
    \section{Ziel der Arbeit}
    Wikipedia hat bewiesen, dass öffentliches Pflegen von Inhalten 
    funktioniert. In dieser Arbeit soll ein Prototyp einer Plattform erstellt
    werden, wo Inhalte zu Medien frei erfasst und bearbeitet werden können.
    Als weitere Neuerung sollen Benutzer die Bewertungsanzeige von Medien auf
    frei definierbare Gruppen, wie zum Beispiel ihre Freunde, festlegen 
    können, um so Bewertungen mit gleichem Geschmack zu sehen.

    Zusätzlich soll auch der Code der Plattform öffentlich verfügbar sein, 
    damit auch dieser von Dritten weiter gepflegt werden kann. Dazu muss die 
    Wahl der Technologien und Tools wohl überlegt sein, damit sich weitere 
    Entwickler finden lassen.
    
    \section{Abgrenzung}
    Die Analysen beschränken sich auf Recherchen im Internet, Büchern und das 
    nähere Umfeld des Studierenden. Es werden keine Umfragen, Erhebungen und 
    Feldstudien durchgeführt. Der Prototyp wird nur über einen Bruchteil der 
    Funktionalität der eigentlichen Plattform verfügen, da es den Rahmen
    der Arbeit sprengen würde.

    \section{Sprache}
    Die Semesterarbeit wurde in deutscher Sprache verfasst. Englische Ausdrücke 
    wurden immer dort verwendet, wo diese im Sprachgebrauch in den verwendeten 
    Programmen genau so gebraucht werden.
    
    Aus Gründen der besseren Lesbarkeit der Semesterarbeit wurde teilweise auf 
    die Nennung beider Geschlechter verzichtet. In diesen Fällen ist die 
    weibliche Form ausdrücklich inbegriffen.
    
    \chapter{Ist Analyse}
    \section{Konkurrenz}
    Welche Plattformen gibt es, was sind ihre USPs.
    \section{Funktionsumfang}
    Übersicht über die Plattformen und ihre Funktionen mit Gewichtung.
    \section{Konklusion}
    Die Schlüsse die aus der Ist Analyse gezogen werden.
    
    \chapter{Konzeption}
    \section{Funktionen}
    Was soll genau Umgesetzt werden.
    \section{Ansichten}
    Skizzierungen der Funktionen.
    \section{Entity Relationship Model}
    ERM Darstellung der Plattform mit obigen Funktionen.
    
    \chapter{Technologien \& Tools}
    \section{Auswahl}
    Welche Technologien kommen für das Konzipierte überhaupt in Frage. Mit
    was für Tools müsste man dazu arbeiten.
    \section{Evaluation}
    Bewertung der Technologien mit Gewichtung auch in Hinblick auf die Tools.
    \section{Konklusion}
    Für welche Technologien udn Projektunterstützende Tools man sich 
    entschieden hat.
    
    \chapter{Prototyp}
    \section{Entwicklungsinfrastruktur}
    \subsection{Entwicklungsumgebung}
    MacOS X, Textmate, Terminal
    \subsection{Versionierungssystem}
    git develop, master
    \subsection{Deployment}
    Systeme: local, staging, production\\
    Datenbanken: sqlite, mysql
    \section{Umsetzung}
    \section{Testen}
    \subsection{Testfälle}
    \subsection{Testresultate}
    
    \chapter{Konklusion}
    \section{Weiterentwicklung}
    \section{Erfahrungsbericht}
    
    \chapter{Installation}
    \section{project zero}
    um project zero zum laufen zu bringen war nötig:\\
    rmv installieren, nokogiri, bundle install
    
    \appendix
    
    \chapter{Personalienblatt}
    \begin{tabbing}
    	\hspace*{4cm}   \= \kill
    	Name, Vorname:  \> {\bf Spross, Silvan} \\
    	Adresse:        \> {\bf Meinrad Lienert-Strasse 27} \\
    	PLZ, Wohnort:   \> {\bf 8003 Zürich} \\
    	\\
    	Geburtsdatum:   \> {\bf 07.11.1985} \\
    	Heimatort:      \> {\bf Zürich ZH} \\
    \end{tabbing}
    
    \chapter{Bestätigung}
    Hiermit bestätige ich, Silvan Spross, dass ich die vorliegende Semesterarbeit
    "`Webapplikation für eine öffentliche Medienbibliothek mit einer API"' im
    Rahmen der geltenden Reglemente selbstständig ausgeführt habe.\\
    \\
    Zürich, den 12. Januar 2011\\
    \\\\
    Silvan Spross
    
    % Abbildungsverzeichnis (kann auch nach dem Inhaltsverzeichnis kommen)
    \listoffigures

    % Tabellenverzeichnis (kann auch nach dem Inhaltsverzeichnis kommen)
    \listoftables
    
    \bibliographystyle{plain}
    \bibliography{literaturverzeichnis}
    
\end{document}